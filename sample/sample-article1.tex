\documentclass[../book.tex]{subfiles}
\begin{document}

\ssrarticle{ぽよぽよを作ってみた}{ぽぽぽ学科 3年}{ぽよ太郎}{ニックネーム的なやつとかtwitter}
\label{14-poyotaro:ch:create-poyopoyo}

\section{poyo}
    セクション

    \subsection{subsection}
        サブセクション

        \subsubsection{subsubsection}
            サブサブサブサブサブキエェーーーーー!!

\section{ソースコード}

    実行可能コード片

    \noindent\begin{minipage}{\textwidth}\mbox{}
\begin{lstlisting}[language=rust,style=code,caption={コードの例}]
vec.shrink_to_fit();  // C++とrustどちらのvectorにもある
\end{lstlisting}%
    \end{minipage}

    定義引用

    \noindent\begin{minipage}{\textwidth}\mbox{}
\begin{lstlisting}[language=c++,style=decleration,caption={宣言の例}]
size_t  hogeFunction(const Piyo &arg1, size_t arg2=0);
\end{lstlisting}%
    \end{minipage}

    \begin{fixme}
        ほげほげについて書く

        \lstinline[language=tex]|begin{fixme}|を使うと、このように完成原稿に入れてはいけないメモを書いておけます。
    \end{fixme}

\end{document}
